\documentclass[a4paper,12pt]{article}
\usepackage[utf8]{inputenc}
\usepackage{graphicx}
\usepackage{amsmath}
\usepackage{geometry}
\usepackage{setspace}
\geometry{left=3cm, right=3cm, top=3cm, bottom=3cm}

% PAGE 1 - Portada

\title{
    \includegraphics[width=0.3\textwidth]{R.jpg}\\
    \vspace{4cm} % Espacio desde el borde superior
    \textbf{\large IMPLEMENTACIÓN DE SISTEMA DE ATENCIÓN DE CLIENTES MEDIANTE 
    TICKETS” ELECTRÓNICO AVANZADO}\\
    \vspace{1cm}
    \textbf{\large MÓDULO INTEGRADOR II}\\
    \vspace{2cm}
    \textbf{\small  Luis Rebolledo Hernández}\\
    \textbf{\small  Pablo Espinoza Gonzalez}\\
    \textbf{\small  Diego Rojas Gonzalez}\\
}

\date{} % Evita que la fecha se muestre

\begin{document}

% Crear la portada
\maketitle

% Insertar una nueva página para el índice
\newpage

% PAGE 2 - Índice
\tableofcontents
\newpage


\section{Descripción del Problema}
En muchas empresas, especialmente en los departamentos de informática y soporte técnico, la gestión de solicitudes de asistencia es ineficiente y desorganizada. Los empleados suelen enviar sus solicitudes a través de correos electrónicos dispersos o llamadas telefónicas, lo que resulta en una falta de seguimiento adecuado y priorización incorrecta de los problemas. Esto lleva a retrasos en la resolución de problemas, pérdida de información importante y una experiencia frustrante tanto para los empleados como para el equipo de soporte. Además, la ausencia de un sistema que aproveche la inteligencia artificial (IA) para automatizar y gestionar de manera más inteligente estas solicitudes agrava el problema.


\section{Planteamiento del objetivo del proyecto}
Optimizar el sistema actual de gestión de tickets utilizado por empresas para la atención de clientes o usuarios que requieran soporte ante problemas que surgen en procesos cotidianos, asegurando una atención rápida y eficiente, canalizando los casos de manera inmediata a las áreas correspondientes para su resolución oportuna.

\section{Justificación}
La mejora del sistema de gestión de tickets es esencial para optimizar la atención al cliente y aumentar la eficiencia operativa. Al canalizar los problemas de manera rápida y adecuada hacia las áreas correspondientes, se reduce el tiempo de respuesta y se mejora la satisfacción del cliente, lo que fortalece la imagen de la empresa y maximiza el uso de los recursos.
\section{Beneficiarios del proyecto}
Los beneficiarios principales del proyecto de mejora del sistema de gestión de tickets serían.

\subsection{Clientes y usuarios}
Serían los más beneficiados al recibir una atención más rápida y eficiente cuando enfrenten problemas o tengan consultas. Esto mejora su experiencia y satisfacción con la empresa.

\subsection{Áreas de soporte y atención}
Los equipos encargados de la atención directa a los usuarios también se verán beneficiados, ya que recibirán los casos de manera más ordenada y podrán enfocarse en resolver los problemas de su competencia específica, reduciendo el desbordamiento de tareas.


\section{Localización del proyecto}
\subsection{Alcance Geográfico}
El sistema de atención mediante tickets tiene un alcance nacional, cubriendo todas las empresas en el país que utilizan este método para servicios de atención al cliente y soporte técnico.

\subsection{Cobertura Territorial}
\subsubsection{Nacional}
El sistema estará disponible en todo el país, con una fase inicial enfocada en grandes ciudades y áreas metropolitanas, antes de extenderse a regiones más rurales.

\subsubsection{Adaptabilidad Regional}
Se ajustará a necesidades locales específicas, como idioma y regulaciones.


\subsection{Población Afectada}
\subsubsection{Empresas}
 Empresas que prestan servicios mediante tickets a nivel nacional.
 \subsubsection{Usuarios}
 Incluye tanto a clientes finales que generan tickets como a empleados internos que los gestionan.

\subsection{Infraestructura de Soporte}
\subsubsection{Centros de Datos}
Servidores distribuidos para alta disponibilidad y seguridad.
\subsubsection{Soporte Técnico}
Equipos de soporte en distintas regiones para atención rápida y efectiva.

\section{Definición de objetivos}
\subsubsection{Objetivo General}
Optimizar la gestión y distribución de tickets en el servicio de atención al cliente a nivel nacional mediante la implementación de un sistema informático avanzado, impulsado mediante la Inteligencia Artificial (IA), con el fin de mejorar la eficiencia, reducir los tiempos de respuesta y aumentar la satisfacción del cliente.
\subsubsection{Objetivos específicos}
\subsubsection{Desarrollar e Implementar el Sistema}
Implementar un sistema informático integral basado en Inteligencia Artificial (IA) para la gestión y distribución de tickets, asegurando su funcionalidad y operatividad en el ámbito nacional.
\subsubsection{Reducir el Tiempo de Respuesta}
Disminuir el tiempo promedio de respuesta a los tickets en un 30 porciento durante los primeros seis meses de operación mediante la automatización y optimización de procesos.
\subsubsection{Mejorar la Satisfacción del Cliente}
Aumentar el índice de satisfacción del cliente en un 25 porciento mediante una gestión más eficiente de los tickets, evaluado a través de encuestas post-servicio.

\subsubsection{Capacitar al Personal de Soporte}
Capacitar al 100 porciento del personal de la empresa en el uso del nuevo sistema antes del lanzamiento, mediante al menos tres sesiones de formación.

\section{Impacto}
La falta de un sistema de tickets centralizado y eficiente, apoyado por IA, tiene varios impactos negativos:

\subsection{Retrasos en la resolución de problemas}
Sin un sistema que priorice automáticamente y asigne tareas según su categoría y criticidad, los problemas críticos pueden pasar desapercibidos, lo que afecta la productividad de toda la empresa.


\subsection{Sobrecarga del equipo de soporte}
El equipo de soporte técnico puede verse abrumado con solicitudes mal documentadas o duplicadas, lo que lleva a un mal uso de los recursos.

\subsection{Satisfacción del cliente interno}
Los empleados pueden sentirse insatisfechos con la velocidad y calidad del soporte que reciben, lo que afecta su moral y rendimiento.

\subsection{Pérdida de datos}
Sin un registro adecuado y automatizado de los tickets, se puede perder información crítica relacionada con la resolución de problemas, dificultando futuras intervenciones y análisis.


\section{Causa Raíz}
La raíz del problema radica en la falta de un sistema unificado y automatizado que utilice IA para la gestión de solicitudes de soporte. La dependencia de métodos informales como correos electrónicos, llamadas telefónicas, o incluso conversaciones en persona, impide la creación de un registro centralizado y estructurado que permita un seguimiento eficiente, priorización automatizada y asignación adecuada de las solicitudes.


\section{Alcance del Problema}
Este problema afecta principalmente a los departamentos de informática y soporte técnico en empresas de cualquier tamaño. Sin embargo, su impacto puede extenderse a toda la organización, ya que la incapacidad para resolver problemas técnicos de manera oportuna puede afectar las operaciones diarias de todos los empleados, independientemente del departamento en el que trabajen.

\section{Implicaciones de No Resolverlo}
Si este problema no se resuelve, las empresas seguirán enfrentando ineficiencias significativas en la gestión de soporte técnico, lo que puede llevar a:
\subsection{Incremento en los costos operativos}
El tiempo adicional necesario para resolver problemas y la necesidad de personal adicional para gestionar la carga de trabajo pueden aumentar los costos operativos

\subsection{Baja productividad}
Los empleados pueden enfrentar tiempos de inactividad prolongados debido a problemas técnicos no resueltos, lo que reduce la productividad general.
\subsection{Desmotivación del personal de soporte}
La falta de un sistema que facilite su trabajo, apoyado por IA, puede llevar a la desmotivación y eventualmente a una alta rotación de personal en el departamento de soporte técnico.
\subsection{Reputación dañada:}
La percepción interna de la efectividad del departamento de informática puede deteriorarse, afectando la confianza en el equipo de soporte técnico.


\section{Criterios de Éxito}
Para considerar exitoso el proyecto de desarrollo del sistema de tickets, se deben cumplir los siguientes criterios:

\subsection{Implementación de un sistema centralizado apoyado por IA}
Todas las solicitudes de asistencia deben ser capturadas, priorizadas y gestionadas en un sistema centralizado y accesible, con la ayuda de IA para automatizar estos procesos.

\subsection{Mejora en los tiempos de respuesta}
El sistema debe permitir un seguimiento eficiente de las solicitudes, resultando en una reducción medible en los tiempos de resolución de problemas.


\subsection{Satisfacción del cliente interno}
Los empleados deben reportar una mejora en la calidad y velocidad del soporte técnico recibido.


\subsection{Análisis y reportes}
El sistema debe proporcionar herramientas de análisis y generación de reportes para identificar tendencias, cargas de trabajo y áreas de mejora en el soporte técnico.

\subsection{Escalabilidad y adaptabilidad}
El sistema debe ser capaz de adaptarse a las necesidades cambiantes de la empresa y escalarse según sea necesario, soportando un crecimiento en el volumen de tickets sin comprometer la eficiencia.

\newpage
\end{document}
